
%%%%%%%%%%%%%%%%%%%%%%%%%%%%%%%%%%%%%%%%%%%%%%%%%%%%%%%%%%%%%
%% HEADER
%%%%%%%%%%%%%%%%%%%%%%%%%%%%%%%%%%%%%%%%%%%%%%%%%%%%%%%%%%%%%

\documentclass[a4paper, onecolumn, oneside, 11pt]{article}
\usepackage{cmbright}
\usepackage[T1]{fontenc}
\usepackage[english]{babel}

%\usepackage[dvips]{graphics} %%graphics and normal LaTeX
%\usepackage{amsmath}
%\usepackage{amsthm}
%\usepackage{amsfonts}
\usepackage{amssymb}
\usepackage{graphicx}
\usepackage{rotating}
\usepackage{pdflscape}
\usepackage{abstract}
\usepackage[ table ]{ xcolor }
\usepackage[square, comma, sort&compress, longnamesfirst]{natbib} %
\usepackage{subfigure}
\usepackage{setspace}
%\singlespacing %% 1-spacing (default)s
%\onehalfspacing

%%% END Article customizations

%%% The "real" document content comes below...

\title{Saccadic Biases}

\author{Alasdair D. F. Clarke, Matthew J. Stainer}


%\date{} % Activate to display a given date or no date (if empty),

% otherwise the current date is printed

\begin{document}

\maketitle

\begin{abstract}
More bias modelling! Cause who can be bothered running actual experiments?


Much effort has been made to attempting to explain eye guidance during natural scene viewing [Tatler, 2009 VIS RES special issue]. However, underlying fixation placement appears to be a set of consistent biases in eye movement behaviour (e.g. see Clarke and Tatler, 2014). We present a model that parametrically accounts for where saccades are directed dependent on where in the bounds of a scene an observer is currently fixating. We find... [some very interesting stuff]. Given that much of our understanding in eye guidance is derived from how people look at pictures on a computer screen, it is important that we use these biases to form a frame upon which to build more sophisticated models of eye guidance that can account for the allocation of gaze above oculomotor behaviours that are independent of the image.

\end{abstract}


\section{Introduction}

Improve on last year's \citep{clarke-tatler2014} effort. More sophisticated biases. And some examples of how to use biases for improved data analysis. 



The human fovea provides a small window of high acuity vision to the world, and as such the locations that we select to view in the world can tell us about how we seek the information necessary to complete the task we are currently undertaking. Current understanding of eye guidance would suggest that fixation locations are selected based on a combination of low-level factors (such as visual salience \citep{Itti:2000te} or orientation information \citep{Baddeley:2006wq}) and high-level factors \citep{Yarbus:1967wd, Buswell:1935tf, Land:2001uc}. However, there are also strong observable biases in eye movement including directional and amplitudinal [XXis that a word?XX] biases in saccades \citep{Tatler:2008uu, Tatler:2009vp, Foulsham:2010fj}, and a strong tendency to fixate near to the centre of images \citep{Tatler:2007hk,Canosa:2003tu}. Importantly, these biases are independent of the viewed content. If we are to gain a complete understanding of the factors that govern eye movement, we must therefore build models of eye guidance on the framework of these underlying biases.


\subsection{The central bias}
There is a strong tendency for people to look close to the centre of pictures \citep{Tatler:2007hk,Tatler:2005bw, Canosa:2003tu, Clarke:2014km} and movies \citep{Tseng:2009jn} presented on computer screens. There have been a number of suggestions for why this might be. One possibility for this effect is that the muscles of the eye show a preference for the `straight ahead' position, re-centring in the orbit of the eye socket for most comfortable contraction of the ocular muscles (an \emph{orbital reserve} \citep{Fuller:1996bx}). As most scene viewing experimental set-ups stabilise the head to increase the accuracy of the eye tracking, and most scenes are presented in the centre of computer displays, such a re-centring mechanism would mean that the centre of images would indeed be preferentially selected. However, when scenes are scrambled into four quadrants, fixations are located near to the centre of each quadrant, rather than the display centre, suggesting that the central tendency is responsive to the viewed content \citep{Stainer:2013ce} rather than the frames of the computer monitor.

Another possibility for the central fixation bias is that it represents a \emph{photographer bias} as photographers tend to frame their shots to include the most important content in the centre of the scene. However, when \cite{Tatler:2007hk} presented scenes where the image features were biased towards the edge of the scene, the central fixation bias persisted. The final possibility is that as a consequence of repeated exposure to photographer bias, the centre of scenes is simply where people are \emph{trained} to look at images \citep{Parkhurst:2002vo}. Such learning of spatial probabilities of targets can explain why, for example, people tend to look around the horizon when searching for people in natural scenes \citep{Birmingham:2009hl, Torralba:2006iq, Ehinger:2009ji}. Expecting to find interesting content in the centre of scenes might be a consequence of this hypothesis typically being correct.

\cite{clarke-tatler2014} revealed that the characteristics of the central bias is remarkably consistent across a series of eye movement databases.... [obviously you are in a better place than me to talk about this paper!]





\subsection{Behavioural biases in saccades}
Further to the observed bias towards the centre of images, it has been revealed that there are underlying biases in the characteristics of eye movement (in terms of the directions and amplitudes of saccades). It has been noted by several researchers that when viewing scenes, there is a higher proportion of eye movements in horizontal directions than vertical or oblique movements (Brandt, 1945; Crundall & Underwood, 1998; Gilchrist & Harvey, 2006; Foulsham, Kingstone & Underwood, 2008; Tatler & Vincent, 2008). There are a number of possibilities as to why this tendency exists (as discussed in Foulsham, Kingstone, & Underwood, 2008). Firstly, there may be a muscular or neural dominance making oculomotor movements in the horizontal directions more likely. Secondly, the characteristics of photographic images may mean that content tends to be arranged horizontally by the photographer. In such situations, horizontal saccades may be the most efficient way to inspect scenes. Thirdly, using horizontal saccades in scene viewing might be a learned strategy. Observers may learn the natural characteristics of scenes based on previous experience, and therefore demonstrate an increased likelihood of moving in the horizontal direction. A final alternative explanation is that this tendency is a consequence of the aspect ratio of visual displays, which normally allow for larger amplitude saccades in the horizontal than vertical directions (Wartburg et al., 2007).

Foulsham and colleagues have presented two interesting exceptions to the horizontal direction bias. Foulsham, Kingstone and Underwood (2008) found that when the orientation of an image is rotated, the distribution of saccade directions follows the orientation of the scene. A second exception comes from using circular apertures (Foulsham & Kingstone, 2010). When a scene is presented in a circular aperture, the tendency to make horizontal saccades disappears, being replaced by a tendency to make vertical saccades relative to the image orientation. However, when using fractal images (where images do not have an obvious orientation), observers tend make horizontal saccades, regardless of the angle that the image is presented.


\subsection{The present study}
The aim of the present study is to characterise the underlying biases of eye movement with which to understand fixation selection in natural scenes. We extend the previous work by examining [XXstuffXX] and 
\section{Methods}

\subsection{Datasets}

\subsection{Pre-processing}
<<<<<<< Updated upstream

\subsubsection{Overall}
Generally, there are some things we want to consider:
\begin{itemize}
\item Normalise fixation positions relative to image frame. 
\item Boot-strapping? 
\item Merge datasets or model individually? 
\item Remove initial fixations? (from all analysis??)
\end{itemize} 

\subsubsection{Saccadic Flow analysis}
Some further preprocessing steps were carried out for the computation of saccadic flow. First of all, we \textit{mirrored} the set of fixations, but adding in reflected copies of the data (reflected in the horizontal, vertical and both midlines). This has two advantages. (i) It is an easy way to make saccadic flow biases in the horizontal or vertical directions. This is similar to how the central bias was defined \cite{clarke-tatler2014}, but by a different mechanism (with the central bias, the model fitting procedure is much simpler and so we just enforced zero mean and 0s in the covariance matrix). (ii) It increases the amount of data available for fitting by a factor of four. This is important as (due to the central bias) there are relatively few saccades that originate from the corners of the images. By equating all corners, we can pool the data and obtain more stable estimates for the underlying distribution. 
=======
Boot-strapping? 
>>>>>>> Stashed changes

For fitting the distributions, we will work in a transformed space. First of all, fixation coords are scaled to $[0,1]$ (relative to image size). Then we do $x' = log(\frac{x}{1-x})$ and $y' = log(\frac{y}{1-y})$. We will only use the transformed space for fitting distributions. All plots and what not will show the data inverse transformed back into normal space $[-1, 1]$.

\section{Biases}

We will model and discuss saccadic flow, coarse-to-fine, and left v right. 

\subsection{Saccadic Flow}

Saccadic flow can be thought of as a generalisation of the central bias. Instead of computing the distribution of all saccadic endpoints in a dataset, we look at the distribution of saccade endpoints given the start points. So for a saccade from $(x_0, y_0)$ to $(x_1, y1)$ we want to model $p(x_1,y_1|x_0, y0)$ This is illustrated in \ref{fig:exampleSaccadic Flow}.



\begin{figure}
%[insert example image here]
\caption{Empirical example of saccadic flow from blah dataset.}
\label{fig:exampleSaccadic Flow}
\end{figure}

\subsubsection{Modelling}

We will model saccadic flow using multivariate skew-$t$ distributions \citep{azzalini2015}. The multivariate skew-normal distribution \citep{azzalini1996} is given by:

\begin{equation}
\phi(z; \lambda) = 2\phi(z)\Phi(\lambda z) 
\end{equation}

for $z \in \mathbb{R}$. I think. 

To characterise how the distribution of saccadic endpoints varies with the start point, we used a sliding window approach. All saccades that originated in a $n\times m$ window were taken and used to fit a distribution. This window was then moved over the stimuli in steps of $s=0.05$  Multivariate polynomial regression was then used to fit 4-th order polynomials to each of the parameters. 

This process was carried out for multivariate normal, skew-normal and skew $t$ distributions. Likelihood of the data given the different distributions can then be compared. 

\subsubsection{Results}

An example of the the distributions vary with saccadic start point is showing in Figure \ref{fig:exampleSkewNormal}.

\begin{figure}
\centering
\subfigure{\includegraphics[width=3cm]{../scripts/flow/flowFigures/saccEndByX1Y4.pdf}}
\subfigure{\includegraphics[width=3cm]{../scripts/flow/flowFigures/saccEndByX2Y4.pdf}}
\subfigure{\includegraphics[width=3cm]{../scripts/flow/flowFigures/saccEndByX3Y4.pdf}}
\subfigure{\includegraphics[width=3cm]{../scripts/flow/flowFigures/saccEndByX4Y4.pdf}}
\subfigure{\includegraphics[width=3cm]{../scripts/flow/flowFigures/saccEndByX1Y3.pdf}}
\subfigure{\includegraphics[width=3cm]{../scripts/flow/flowFigures/saccEndByX2Y3.pdf}}
\subfigure{\includegraphics[width=3cm]{../scripts/flow/flowFigures/saccEndByX3Y3.pdf}}
\subfigure{\includegraphics[width=3cm]{../scripts/flow/flowFigures/saccEndByX4Y3.pdf}}
\subfigure{\includegraphics[width=3cm]{../scripts/flow/flowFigures/saccEndByX1Y2.pdf}}
\subfigure{\includegraphics[width=3cm]{../scripts/flow/flowFigures/saccEndByX2Y2.pdf}}
\subfigure{\includegraphics[width=3cm]{../scripts/flow/flowFigures/saccEndByX3Y2.pdf}}
\subfigure{\includegraphics[width=3cm]{../scripts/flow/flowFigures/saccEndByX4Y2.pdf}}
\subfigure{\includegraphics[width=3cm]{../scripts/flow/flowFigures/saccEndByX1Y1.pdf}}
\subfigure{\includegraphics[width=3cm]{../scripts/flow/flowFigures/saccEndByX2Y1.pdf}}
\subfigure{\includegraphics[width=3cm]{../scripts/flow/flowFigures/saccEndByX3Y1.pdf}}
\subfigure{\includegraphics[width=3cm]{../scripts/flow/flowFigures/saccEndByX4Y1.pdf}}
\caption{Multivariate skew-$t$ distributions fitted to fixation location, by saccade start point.}
\label{fig:exampleSkewNormal}
\end{figure}

Figure \ref{fig:smParamsOverSpace} shows how the parameters for the skew-$t$ distribution vary over horizontal position for a selection of vertical positions. The regression coefficients are given in Table 



\begin{figure}
\centering
\includegraphics[width=13cm]{../scripts/flow/paramsChagingOverSpace.pdf}
\caption{Multivariate skew-normal parameters over space.}
\label{fig:smParamsOverSpace}
\end{figure}


\begin{table}
\centering

\begin{tabular}{c c}

parameter & equation \\
\hline
$\Omega_{x,x}$	& $= 0.33+ 0.38x^2 -0.29y^2 + 0.02x^4 + 0.22y^4$ \\ 
$\Omega_{x,y}$	& $=x + y + x^2 + y^2 + x^3 + y^3 + x^4 + y^4$ \\ 
$\Omega_{y,x}$	& $=x + y + x^2 + y^2 + x^3 + y^3 + x^4 + y^4$ \\ 
$\Omega_{y,y}$	& $=x + y + x^2 + y^2 + x^3 + y^3 + x^4 + y^4$ \\ 
\hline
$\alpha_{x^2}$		& $=x + y + x^2 + y^2 + x^3 + y^3 + x^4 + y^4$ \\ 
$\alpha_{y^2}$		& $=x + y + x^2 + y^2 + x^3 + y^3 + x^4 + y^4$ \\ 
\hline
$\nu$			& $=x + y + x^2 + y^2 + x^3 + y^3 + x^4 + y^4$ \\ 
\end{tabular}

\caption{Parameter model - clearly I still have to fill in all the coefficients!}
\label{tab:paramModel}
\end{table}

\subsubsection{Discussion}



\subsection{Coarse-to-fine}

People make shorter saccades over time. Include $1/f$ dynamics? 

\subsection{Left v Right}

Initially more fixations to the left half of the image \citep{nuthmann-matthias2014}.

\section{Using Biases for Better Analysis}

We will use the the central bias \citep{clarke-tatler2014} and \textit{saccadic flow} in some different contexts to see what biases can do for vision research. :p


\subsection{Attentional Landscapes}

HELLO THERE MY FRIEND

\begin{figure}
\includegraphics[width=\textwidth]{../scripts/heatmaps/centreadjustedheatmaps.pdf}
\caption{Traditional 'heat map' plots of fixations normalised by the central bias. This method allows us to characterise fixations that are less accountable for by image-independent central biases.}
\label{fig:centreAdjustedHeatmaps}
\end{figure}



Or do we call them hotspot maps?

\subsection{ROC Analysis}

Example of using our models rather than shuffle approaches.

\subsection{Flow and Coarse to fine}
To what extent does saccadic flow account for coarse-to-fine dynamics

\subsection{Inverse Yarbus}

Do these biases allow us to improve inverse yarbus performance?

\subsection{Salience}

Does salience explain the less likely saccades? 

\section{Discussion}



\subsection{Scenes and natural viewing behaviour}
That observers organise their viewing behaviour on computer screens around the reference frames provided by the bounds of scenes (see also \cite{Stainer:2013ce}) causes problems for relating findings of eye guidance in scenes to eye guidance in natural behaviour, as the bounds of such reference frames are unclear in the real world. While it has been suggested that we tend to fixate near to the centre of our `straight ahead' head position [FOULSHAM WALKING, CRISTINO AND BADDELEY?], there are no discrete edges as are typical in computer based scene viewing paradigms. If fixation locations are constrained by the bounds of the scene, this highlights the care we must take about the generalisations we make from findings in the lab to the real world (see [kingstonepaper 2010]). 






\section*{Acknowledgements}

Thanks to Adelchi Azzalini for advice on using the \texttt{sn} package for \texttt{R}. And mention grants. 

\bibliographystyle{plainnat}
\small
\bibliography{literature}
\end{document}